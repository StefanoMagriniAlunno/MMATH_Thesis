\chapter{Introduction}
%Introduco la tesi magistrale.
\begin{toDo}
	Idee su dei contenuti:
	\begin{itemize}
		\item Introduzione
		\begin{itemize}
			\item Perché è importante confrontare opere grafiche?
			\item Quali sono le applicazioni pratiche?
			\item Quali problemi o limitazioni esistevano in precedenza?
		\end{itemize}
		\item Genesi
		\begin{itemize}
			\item In breve la tesi triennale
			\item Problemi e limiti
			\item Punto chiave: cosa succede se lavoro con molti colori?
			\item Idee e tentativi proposti
			\item Come cambia la tesi
		\end{itemize}
		\item Conclusione
		\begin{itemize}
			\item Raccolta del dataset
			\item Pulizia del dataset
			\item Realizzazione di frameworks
			\item Analisi dei risultati
			\item Ricerca teorica
		\end{itemize}
		\item Organizzazione della tesi
		\begin{itemize}
			\item Literature Review: Argomenti portati
			\item Methodology: scoperte notevoli
			\item Results: risultati notevoli
		\end{itemize}
	\end{itemize}
\end{toDo}

\begin{toReview}
	\section{Attribution of handwriting works}
		\noindent The comparison of graphic works plays a very important role in several fields, including art history, digital forensics and intellectual property protection. By analysing the characteristics of graphic works, it is possible to identify the author, verify the authenticity of a work or detect possible counterfeits. In art history, for example, stylistic and technical analysis of handwritten notes or sketches can provide valuable insights into the creative processes of renowned authors. Similarly, in digital forensics, the comparison of graphic works can help detect forged documents or identify alterations to legal documents.

		\noindent Beyond these practical applications, the ability to compare graphic works also opens up possibilities for understanding more comprehensive patterns. For example, it can help uncover stylistic influences between artists or identify recurring patterns within a collection. In the context of machine learning and data analysis, graphic comparison serves as a basis for developing algorithms capable of processing complex visual data, which is increasingly important in an era dominated by digital media.

		\noindent However, the process is not without its challenges. The presence of noise, variations in resolution and the diversity of graphical styles make it difficult to establish a robust and reliable framework for comparison. This thesis aims to address these issues by developing methodologies that improve the accuracy and adaptability of graphic work comparisons.

		\bigskip
		\noindent The practical applications of comparing graphic works cover a wide range of fields, each of which benefits from customised analysis techniques:

		\begin{itemize}
			\item \textbf{Authorship attribution}: Determining the author of a handwritten document or artistic work is important in fields such as art history, where verifying the authenticity of an artist can have a significant impact on the cultural and financial value of the work.
			\item \textbf{Forgery Detection}: In digital forensics and legal investigations, identifying alterations to documents or detecting forgeries in graphic works plays a key role in ensuring authenticity and legality.
			\item \textbf{Intellectual Property Protection}: The ability to compare graphic works is critical for enforcing copyright laws and resolving disputes over original creations.
			\item \textbf{Historical Analysis}: In the study of historical documents and manuscripts, graphic comparison helps to trace stylistic influences, identify authors, and reconstruct fragmented works.
			\item \textbf{Digital Archiving and Restoration}: Automated comparison methods help to cluster, catalogue and restore large collections of graphic works, ensuring their preservation for future generations.
			\item \textbf{Educational Tools}: Comparison of graphic works can also be used in educational contexts, providing automated feedback on handwriting or artistic style for students and professionals.
		\end{itemize}

		\noindent These applications demonstrate the versatility and importance of robust graphic comparison methods. Each context presents unique challenges, such as the need to deal with different resolutions, styles and noise levels, which this thesis aims to address through innovative methods.

		\bigskip
		\noindent Despite its importance, the comparison of graphic works faces several challenges and limitations that have hindered progress in this field:

		\begin{itemize}
			\item \textbf{Distortions and impurities}: Graphic works, especially handwritten or historical documents, often contain noise such as background patterns, stains or scanning distortions. These contaminants can distort the analysis and reduce the reliability of the comparison results.
			\item \textbf{Variability in resolutions and formats}: Works are often digitised at different resolutions and stored in different formats, making it difficult to standardise data for analysis. This variability makes it difficult to extract meaningful features.
			\item \textbf{High dimensionality and computational cost}: Graphic works are represented as high-dimensional data, especially when detailed features or pixel-level analysis are involved. This increases computational costs and limits the feasibility of large-scale comparisons.
			\item \textbf{Limited robustness of clustering techniques}: Traditional clustering methods, such as hard \gls{kmeans}, struggle with noisy and overlapping data distributions, leading to suboptimal results in many real-world scenarios.
			\item \textbf{Lack of standardised datasets}: The lack of well-curated and representative datasets for testing and validating comparative methods makes it difficult to benchmark algorithms and ensure their generalisability.
			\item \textbf{Subjective Preprocessing Steps}: Many preprocessing techniques depend on manual adjustments or heuristics, which can introduce bias and limit the reproducibility of the analysis.
		\end{itemize}

		\noindent These issues highlight the need for advanced methods that can adapt to noise, handle different data representations, and provide reliable results in a range of scenarios. This thesis directly addresses these challenges by refining preprocessing techniques, introducing fuzzy clustering for improved robustness, and exploring scalable solutions for high-dimensional data.

	\section{Background of the project}

\end{toReview}
