\begin{abstract}
The attribution of authorship in graphic works is a topic of interest in various fields, including art history, intellectual property protection and digital forensics. This thesis represents a continuation of my previous thesis, which proposed a pioneering method for authorship attribution based on a simplified representation of images. The aim was to address some of the more obvious limitations of the previous methodology, such as the inflexibility of pre-processing and the need to work with binary images.

\noindent In this paper, a new application based on fuzzy clustering techniques (FCM) is introduced. This application enables the analysis of greyscale images, improving the model's ability to adapt to complex and noisy data. Although the method still has significant limitations in terms of reliability compared to more established techniques, it represents a first step towards an idea that, if developed further, could offer an interesting alternative.

\noindent The methodology was applied to a dataset of $113$ sheets of university notebooks, with results showing a false negative rate of $6\%$ and a false positive rate of $18\%$. Although these results are not perfect, they highlight the potential of the proposed method and suggest directions for future investigation.

\noindent It is important to note that the intention of this paper is not to replace traditional methods, but rather to propose a complementary avenue that, with further development, could enhance the existing range of authorial attribution techniques.
\end{abstract}
