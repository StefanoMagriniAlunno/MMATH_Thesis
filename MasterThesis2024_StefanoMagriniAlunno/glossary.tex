\newglossaryentry{boost}
{
    name=boost,
    description={In computer science, "boost" generally refers to enhancing the performance of a system, application, or algorithm through code optimization, leveraging more efficient hardware resources, or implementing advanced techniques}
}
\newglossaryentry{thread}
{
    name=thread,
    description={A thread is a fundamental unit of execution in computing, allowing a program to perform multiple tasks concurrently, akin to different lines of thought in a mathematical problem-solving process}
}
\newglossaryentry{warp}
{
    name=warp,
    description={A warp is a group of threads executed together in a synchronized manner within a GPU, allowing for efficient parallel processing of data}
}
\newglossaryentry{kernel}
{
    name=kernel,
    description={A kernel in GPU programming refers to a small program or function that is executed in parallel across multiple threads on a GPU, typically used to perform computations on large datasets or to accelerate specific tasks}
}
\newglossaryentry{Python}
{
    name=Python,
    description={A high-level, general-purpose programming language}
}
\newglossaryentry{CUDA}
{
    name=CUDA,
    description={Dialect of the C++ language for handling the video card}
}
\newglossaryentry{dataset}
{
    name=dataset,
    description={A data set is a collection of data}
}
\newglossaryentry{png}
{
    name=png,
    description={The PNG (Portable Network Graphics) format is a raster image file format used for lossless compression. This means that images saved in this format do not lose quality or detail during compression}
}
\newglossaryentry{kmeans}
{
    name=KMeans,
    description={K-means is an unsupervised clustering algorithm that divides a data set into k distinct groups based on distance. Each group is defined by its centroid, and the goal of the algorithm is to minimise the variance within the groups}
}
\newglossaryentry{cxx}
{
    name=C++,
    description={Un linguaggio informatico a scopi generici largamente usato per realizzare codice ad alte prestazioni per sistemi embedded}
}
\newglossaryentry{thrust}
{
    name=thrust,
    description={libreria di C++ che può gestire anche dati attraverso il processore grafico}
}
\newglossaryentry{cuBLAS}
{
    name=cuBLAS,
    description={libreria di CUDA-Toolkit specializzata nel calcolo lineare e gestione di dati su processori grafici}
}
\newglossaryentry{cursedim}
{
name=curse of dimensionality,
description={The curse of dimensionality refers to various phenomena that arise when analyzing and organizing data in high-dimensional spaces that do not occur in low-dimensional settings}
}
\newglossaryentry{montecarlo}
{
	name=Monte Carlo method,
	description={Monte Carlo methods, or Monte Carlo experiments, are a broad class of computational algorithms that rely on repeated random sampling to obtain numerical results}
}
\newglossaryentry{r}
{
name=R,
description={Un linguaggio informatico usato per lo più per analisi statistiche. Include non solo molte funzionalità ma anche dataset.}
}

% sigle che fanno riferimento al progetto stesso
\newacronym{dada}{DADA}{Discrete Automatic Drawings' Analysis}
\newacronym{cada}{CADA}{Continuous Automatic Drawings' Analysis}

% sigle per termini informatici
\newacronym{gpu}{GPU}{Graphics Processing Unit}
\newacronym{cpu}{CPU}{Central Processing Unit}
\newacronym{rgb}{RGB}{Red Green Blue}
\newacronym{hsl}{HSL}{Hue Saturation Lightness}
\newacronym{gpgpu}{GPGPU}{General-Purpose computing on Graphics Processing Units}
\newacronym{sm}{SM}{Stream Multi-Processing}
\newacronym{hpc}{HPC}{High Performance Computing}
\newacronym{cuda}{CUDA}{Compute Unified Device Architecture}

% sigle per termini tecnici generici
\newacronym{dpi}{DPI}{dots per inch}

% sigle matematiche
\newacronym{gmm}{GMM}{Gaussian Mixture Models}
\newacronym{svd}{SVD}{Singular Value Decomposition}
\newacronym{fft}{FFT}{Fast Fourier Transform}
\newacronym{dft}{DFT}{Discrete Fourier Transform}
\newacronym{cft}{CFT}{Continue Fourier Transform}
\newacronym{idft}{IDFT}{Inverse Discrete Fourier Transform}

% sigle per termini teorici
\newacronym{fcm}{FCM}{Fuzzy C-Means Clustering}
\newacronym{em}{EM}{Expectation-Maximisation}
\newacronym{map}{MAP}{Maximum A Posteriori}
